\section{Conclusion}
The goal of our project was to design a sensor device that could monitor the concentration of various pollutants. The sensor node would be able to transmit this information using a LoRaWAN network to an external server where it could be viewed by users over the internet.

In order to meet this design goal, we first came together as a group to determine a set of requirements. Out of all the requirements we determined, the three most important ones that we established were that the sensor node could transmit sensor measurements over a LoRaWAN network where they could be viewed in a graphical and textual format, the server will send periodic air quality reports and notifications to the user via Twitter, SMS, and/or email, and that the sensor node will be able to dynamically power the node from either battery, solar, or USB. 

We knew that to create a design that met these requirements, we would need to perform more research in variety of areas. One of these areas we spent a lot of time researching was in the LoRaWAN standard itself. We familiarized ourselves with the benefits and limitations of the technology. Another key research area for us was in air pollution. Given that one of our design goals was to report on air quality and different air pollutants, we knew we would have to learn more about the various types of air pollution to know what types of sensors to purchase. Finally, we spent the majority of our research time on picking the components of our design. While not all components were fully chosen before working on the design, major components such as the LoRa networking and compute module and gas sensors were critical and the design was based around them, so these components were decided upon early.

After performing research, we determined that we would require three main components for our project to function: a sensor node, a LoRaWAN gateway, and a web server. Due to time and economic constraints, we decided to focus our primary design efforts on the sensor node. This would mean we would create a custom PCB and write all the embedded software for the node. For the gateway, we decided to use off-the-shelf components. Those components being a Raspberry Pi and a RAK 2245 LoRaWAN gateway module. We would also use open source software for running on the gateway. For the server, we would use the free-tier version of an Amazon Web Services instance running AWS IoT Core. We would also design the web application that would display data collected from the sensors and that the user would interact with.

Next, we discussed what we are doing and what our plans our to prototype our design as well as test it. Currently, we have purchased a development board with the same LoRa module that we are using in our final design in order to prototype the embedded software and test sending data to the server. This is the section where we also include the bill of materials based on our current schematic. We also have laid out a plan for testing all of the components in our design. This includes power components, such as batteries and power ICs, all of our sensors, and the microcontroller. We have also detailed our plans for testing all of the software in our project. This includes the software running on our embedded microcontroller and the software that will be running on our web server.

Overall, after performing the necessary to write this paper, we feel that we are well on track with our design. We plan to continue working on this project into the Christmas break. Our next plans include connecting our development boards to our AWS server and sending some test data to display on a basic website. We also hope to order some initial PCBs based on our current schematics. Our team is optimistic with how the project is progressing and we are all excited continue the work of implementing our design.