\subsection{LoRa-E5 MCU}
The ultra-low power LoRa-E5 (STM32WLE5JC) will be the microcontroller responsible for providing
long-range wireless communication from the nodes to the gateway. It's a Ultra-Thin Profile Fine
Pitch Ball Grid Array (UFBGA) with 73 ball arrays and a 28 pin layout. The chip is powered through
VCC Pin 1 which takes 3.3V from the main power rail. Pins PB15 and PA15 are the microcontroller's
serial clock and serial data I2C pins which are used to communicate with the sensors in the system.
The respective pins on each sensor will be connected to PB15 and PA15 so sensor data can be read at
defined intervals. Pin 15 is the RFIO pin which will be connected to the antenna. Additionally,
3 GPIO pins were used to read the 3 status states of the battery charge controller. Pin 21 (PA9) is
connected to the MCP73871 PG pin which indicates the system is receiving power. GPIO pin 13 (PC0) on
the microcontroller is connected to pin 8 (STAT1/LBO) on the MCP73871 which indicates to the
microcontroller when the battery output is low or the system is charging. The third GPIO used is pin
20 (PB10) on the microcontroller which is connected to pin 7 (STAT2) on the MCP73871 which indicates
to the microcontroller when the battery has been fully  charged. The nodes will have LEDs to
indicate battery status to the users in addition to having the ability to remotely monitor battery
status through the LoRa microcontroller. 

\subsubsection{Memory and DMA}
The STM32WLE5JC's flash memory controller features 256 KB of flash with 72-bit read and write bus (8
being for ECC (Error Correctly Control). It also has 64 KB of SRAM with 32 instruction caches lines
of 256 bits (1 KB in total) and 8 data cache lines of 256 bits (256 Bytes in total). The upper 31 KB
are reserved for system information, option bits, and system memory. The system memory, the largest
section of 28 KB, is used to store the boot loader, which is capable to reprogram the flash memory
through USART, \iic, or SPI. The region can be protected through to avoid anyone from being able to
be able to write to it. The ability for the MCU to load the bootloader into RAM and use that to
reprogram the flash will be important to satisfying one of our stretch goals of being able to do
remote firmware upgrades \cite{ds-stm32wle5j8}.

% The sys

\subsubsection{Hardware Semaphores}
Hardware support for semaphores is key for synchronization amongst multiple threads, such as when
using FreeRTOS since it relies on having at least two threads: idle and a single application thread,
as described in Section \ref{sec-freertos}. The STM32WLE has 16 32-bit hardware semaphores, and can
be locked using the core ID (COREID) and process ID (PROCID). Once locked, they can only be unlocked
with a matching COREID and PROCID \cite{ds-stm32wle5j8}. Semaphores will be heavily used in our
design for allocating resources (binary semaphores) and signalling threads.

\subsubsection{Clock Sources}
The microcontroller also has support for a real-time-clock (RTC), which is important for
timestamping LoRaWAN packets, required by the LoRaWAN protocol and for location services. The RTC
can be used to wake up the devices based on program intervals or at a programmed time. The RTC never
stops, even in the lowest power mode, standby. The RTC will be used in the design for waking up the
microcontroller from the low-power standby mode. The RTC gets its clock source from an external
32.768 KHz crystal, referred to as the LSE (low-speed external) clock. The other clock, the HSE,
(high-speed external) clock, oscillates at 32 MHz, and is used to clock the rest of the chip
including the RF module. The RF module is sourced from an internal RC (resistor-capacitor) clock
derived from the HSE clock, and oscillates at 13 MHz. During sleep modes, the RF module can opt for
the 64 KHz oscillator. Other clock sources include the LSI (low-speed internal), MSI (multi-speed
internal - a configurable RC clock), HSI (high-speed internal) 16 MHz clock, and the PLL
(phased-lock loop) configurable clock. Several peripherals have specific requirements about which
clock sources they can use. As for the MCU core, the core can accept any of the previously mentioned
clock sources, but is limited to which source it can use depending on the run mode it's currently
operating in. For example, sleep mode (stop mode 0) can only use the HSI clock \cite{ds-stm32wle5j8}.

\subsection{Low Power Modes}
Switching between different system operating states will be important to extend the microcontrollers
battery life. We have done multiple case studies to show how important it will be to not only keep
the sleep power low, but to shorten periods of high power operation, such as when operating the
particulate matter sensor or transmitted data over LoRa (the most costly). The microcontroller
supports many low power modes with varying power usages and limitations. Table
\ref{tab:mcu-lp-modes} summarizes the low power modes detailed in the data sheet
\cite{ds-stm32wle5j8}. In all cases, the core clock is off. The current draw was the max current
draw at 25 degrees Celsius at 3.0V, and the max clock speed for that mode.

\begin{table}[H]
  \centering\footnotesize
  \caption{The STM32WLE5JC's Low Power Modes}
  \begin{tabularx}{\linewidth}{| p{0.13\linewidth} | p{0.13\linewidth}|p{0.13\linewidth}|X|}
    \hline
    Power Mode & Max Clock (MHz) & Current Draw (\uA) & Peripheral Limitations 
    \\\hline\hline

    Sleep 
    & HSE
    & 2100
    & None.
    \\\hline

    LPRun 
    & 2 MHz
    & 270
    & PLL disabled, lower RF module power, low-power voltage regulator used for main power.
    \\\hline

    LPSleep 
    & 2 MHz
    & 95
    & Same as LPRun, core disabled.
    \\\hline

    Stop 0 
    & HSI16 
    & 540
    & PLL, MSI, HSI16, and HSI32 are disabled. Peripherals not supporting wakeup are power gated. Wakeup capable use HSI on request.
    \\\hline

    Stop 1 
    & HSI16 
    & 6.40
    & Same as Stop 0, but with slower wakeup time.
    \\\hline

    Stop 2 
    & HSI16 
    & 1.40
    & Core power domain is disabled. Only some peripherals keep their state. The radio can remain
    on.
    \\\hline

    Standby (with SRAM2 and RTC) 
    & MSI 4 MHz
    & 0.345
    & Disabled voltage regulators, SRAM1 off, only can use RTC and external pin interrupts.
    \\\hline

    Standby (with RTC)
    & MSI 4 MHz
    & 0.260
    & All SRAM contents lost. Otherwise same as Standby (with SRAM2).
    \\\hline

  \end{tabularx}
  
  \label{tab:mcu-lp-modes}
\end{table}

\subsubsection{I/O Peripherals}
In our design, we will using many of the peripherals available on the microcontroller to provide the
functionality we need. The following table summarizes the functionalities of the different
peripherals we will be using in our design, along with the section we detail its functionality.

\begin{table}[H]
  \centering\footnotesize
  \caption{Used MCU Peripherals}
  \begin{tabularx}{\linewidth}{| p{0.13\linewidth} | X |p{0.13\linewidth}|}
    \hline
    Peripheral & Usage & Section
    \\\hline\hline

    GPIO
    & LED status, power subsystem status and control, reset, USB external wakeup
    & \ref{sec:power}, \ref{sec:sys-io}
    \\\hline

    Sub-GHz RF
    & LoRa and LoRaWAN
    & \ref{sec:embedded-software}
    \\\hline

    \iic
    & Sensor communication
    & \ref{sec:sensing}
    \\\hline

    UART
    & Communication over USB
    & \ref{sec:sys-io}
    \\\hline

    ADC
    & Battery capacity monitoring
    & \ref{sec:power}
    \\\hline

  \end{tabularx}
  
  \label{tab:mcu-peripheral-usage}
\end{table}
