\subsection{The Backend Software Stack}
The backend software stack will be responsible for collecting data from the
sensor nodes and displaying the air quality data on webpage. It will also be
responsible for sending alerts via text or on Twitter about poor air quality
conditions. It will be the main way users will interact with their deployed
sensors and view data. The backend will also allow users to manage, configure,
and send commands to the sensor nodes remotely. The backend is composed of three
main components: (1) the LoRaWAN gateways running the LoRaWAN Network Server,
(2) a server running on an IoT host provider that can collate data from the
gateways and provide other backend services, and (3) a web user interface
running on the IoT backend provider (such as AWS IoT or Azure IoT) that displays
statistics and user specific information.

The sensor nodes communicate over LoRa using the LoRaWAN protocol to gateways.
These gateways are running what is called the LoRaWAN Network Server (LNS),
which is a standardized protocol for connecting devices using LoRaWAN to the
internet.  LoRaWAN network servers are the bridge between LoRa and the Internet.
LoRa network servers can be public or private. As previously mentioned in the
LoRaWAN standardization section, devices communicating over LoRaWAN perform a
handshake, and exchange a set of IDs that identify the device, the join server,
and the application/network session. These IDs are: DevEUI, JoinEUI (known
previously as the AppEUI), and AppKey.

\subsubsection{The Chirpstack Network Server Stack}
\subsubsection{The Things Network}
\subsubsection{The Helium Network}
\subsubsection{Other Private Networks}
\subsubsection{Microsoft Azure IoT}
\subsubsection{AWS IoT}
