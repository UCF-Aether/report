\subsection{Analysis of Similar Existing Products}
In this section, we will provide a brief survey similar existing AQI monitoring/pollution monitoring nodes. We used what we learned from exploring these existing solutions in order to provide some guidance as to what we should strive towards in terms of design requirements. By analyzing these existing devices, we can better determine where our project should be in terms of factors, such as cost, battery life, and feature set.

\subsubsection{MClimate AQI Sensor and Notifier LoRaWAN}
The MClimate AQI Sensor and Notifier LoRaWAN is an indoor AQI sensor. The device is availble on MClimate's website for \$146. It is described as being handle 10+ years of battery life. It also features a sensor from Bosch that is able to detect VOCs, temperature, humidity, and barometric pressure \cite{mclimate-aqi-sensor}. The specific sensor model is not listed on the datasheet, however, it seems to be similar to the one we are planning on incorporating into our design. With regards to the specifics of the LoRa transceiver, the device transmits at 14 dB and has a link budget of 130 dB. The device communicates data over LoRaWAN to a server owned by the company. This data is then able to be viewed via a mobile app. In addition, the device features an audible alarm and visual LED indicator to warn users if the AQI drops below a determined threshold. The listed physical dimensions are 80 x 80 x 19 mm. The listed weight is 68 grams.

The device is similar to our proposed design in some ways, but there are some key differences. A major advantage of our design is the inclusion of a particulate matter sensor. While the MClimate AQI Sensor is marketed as indoor air quality sensor and particulate matter is not issue in that environment, the addition of a particulate matter sensor allows our design to function outdoors as well. Another benefit of our design is the increased transmit power that the LoRa module we selected supports.

\subsubsection{Davis Instruments AirLink}
The Davis Instruments AirLink is an air quality monitoring device that supports measuring indoor and outdoor air quality. It features a particulate matter sensor that can perform measurements of PM1.0, PM2.5, and PM10. Additionally, the device measures temperature, humidity, dew point, heat index, and wet bulb. The device also supports sending data to a mobile app and website. The device does not use LoRaWAN and instead uses WiFi. The device does not support the use of a battery, meaning it must be connected to a power source with a 5 V DC-AC adapter. The AirLink supports both indoor and outdoor operation. To support this, the AirLink includes a weatherproof cover that can be placed over the device to protect it from the outdoor elements. The listed physical dimensions without the weather cover are 2 x 3.5 x 1 in and the dimensions with the cover are 4 x 4.5 x 1.5 in. The weight of the device without the cover is 3.7 oz and the weight of the device with the weather cover is 6.5 oz \cite{davis-airlink}.

Compared to our proposed design, this device has a somewhat similar feature set. Both our design and the AirLink support particulate matter measurement, as well as all of the other various weather measurements. However, a there exists a major difference between the portability and flexibility of the AirLink and our proposed design. Due to the fact that the AirLink only supports WiFi and must be connected directly to a wall outlet makes this device have a very small range. Taking advantage of the increased communication range provided by using LoRaWAN allows our design to be placed almost anywhere. In addition, having a battery makes our design able to be placed in more locations and also makes it easier to install.