\subsection{The Firmware Test Plan}
The firmware testing will utilize our existing CMake build system the Unity unit
testing library. Tests will be split into two types: target-agnostic and
target-specific. Target agnostic tests are tests that can run on any
architecture that can compile C. They are tests that don't reach down
into the microcontrollers HAL, low-level drivers, or hardware specific
peripherals or registers. Target-specific tests are tests that use hardware
specific features, registers, or peripherals. These tests might check the status
of the LoRa RF module, correct operation of \iic, or the correct operation of
the sensors. Target-agnostic tests will be configured to run in a GitHub
continuous integration pipeline to help automate and ease development of new
features. Target-specific tests will have to be run manually. Since we are using
FreeRTOS, we will need to create tests that can check for deadlocks, thread
starvation, and other issues commonly had in concurrent programming. The
firmware will also need to be error-tolerant, and correctly recover from as many
errors as possible. Issues will be tracked using ClickUp or GitHub, and assigned
to someone to solve.

Unit tests will only be written for key subsystems. Having too many unit tests
creates a burden for the programmer, but just enough keeps the code base mostly
free of certain types of errors. The general flow for developing and testing the
firmware will be the following: first write the code and consider the downward
effects of the code having issues and determine if it's a critical system that
needs to be guarded by unit tests. After writing the unit tests, on each
iteration of the firmware, the code will be automatically tested when pushed to
GitHub using the target-agnostic tests using GitHub Actions. Before creating the
next major version of the firmware, the target-specific tests will be run to
ensure its functionality. If all tests are passing, the code will be merged into
the main branch, and deployed either manually or automatically through LoRa OTA
updates. Then process then repeats until the firmware is complete or
near-complete.

\subsubsection{Unity}
Unity is a unit testing framework written for C with embedded systems in mind.
It's able to run on 8 or 64 bit microcontrollers due to its simple design and
small footprint. It has easy CMake integration to allow for automated building
and testing. Unity has less than 3000 lines of code, including white space.
Unity provides a set of assertions to test various aspects of the firmware. They
can be simple assertions, such as:

\begin{lstlisting}[language=C, caption=Unity Basic Assertion Example
\cite{unity-homepage}]
int a = 1;
TEST_ASSERT( a == 1 ); //this one will pass
TEST_ASSERT( a == 2 ); //this one will fail
\end{lstlisting}

Unity also provides hundreds of specialized assertion definitions, such as
printing a message on failure, testing if an integer is within a certain range,
or testing each int or float in an array.

\begin{lstlisting}[language=c, caption=More advanced Unity Assertions
\cite{unity-github}]
TEST_ASSERT_EQUAL_INT(2, a);
TEST_ASSERT_EQUAL_HEX8(5, a);
TEST_ASSERT_EQUAL_UINT16(0x8000, a);
TEST_ASSERT_EACH_EQUAL_INT(expected, actual, num_elements);
TEST_ASSERT_MESSAGE( a == 2 , "a isn't 2, end of the world!");
\end{lstlisting}

After generating the make files with CMake, compiling the tests, and either
running on the local machine for target-agnostic tests or on the development
board for target-specific tests, Unity will print a report of the test results.
The report contains the number of failing and passing tests, along with where
exactly the tests failed. All example code has been gathered from either Unity's
homepage \cite{unity-homepage} or the source code on GitHub \cite{unity-github}.
Below is some sample output:

\begin{lstlisting}[caption=Example Unity Test Report Output \cite{unity-homepage}]
testUnit1.c:21:test_AverageThreeBytes_should_AverageMidRangeValues:PASS
testUnit1.c:22:test_AverageThreeBytes_should_AverageHighValues:PASS
-----------------------
2 Tests 0 Failures 0 Ignored
OK
\end{lstlisting}
